\documentclass{article}
\usepackage{authblk}
\usepackage{indentfirst}

\title{Eulerian Strom Track}
\author[1]{Jeyavinoth Jeyaratnam}
\author[1]{James F. Booth}
\affil[1]{The City College of New York, CUNY, New York}

\begin{document}
\maketitle

\noindent
By filtering atmospheric data temporally, in a manner that removes the diurnal and the greater than weekly variability, one can isolate the synoptic variability (Blackmon et al. 1976). Then, the standard deviation of the filtered data at each latitude and longitude can be interpreted as the climatological baroclinic wave activity, which, for historical reasons, is termed storm tracks (Wallace et al. 1988). The storm tracks give a simple large-scale metric for the skill in the model representation of extratropical cyclones, in terms of location of the storms, their seasonality and their intensity, which correlates very strongly with transient poleward energy transport.

\hfill

\noindent
To isolate the synoptic timescale, this algorithm uses 24-hour differences of daily-averaged data. Using daily averages removes the diurnal cycle and the 24-hour differencing removes variability beyond 5 days (Wallace et al. 1988). After filtering the data to create anomalies, the variance of the anomalies is calculated across the four seasons for each year. Then the seasonal variances are averaged across all years. For the first year in the sequence, the variance for JF is calculated and treated as the first DJF instance. For the final December in the sequence is not used in the calculation.


\section*{Version \& Contact info}

\noindent
Version 0.1 :: May 6$^{th}$, 2014

\noindent
{\bf Current Developer:} Jeyavinoth Jeyaratnam (jjeyaratnam@ccny.cuny.edu), CUNY

\noindent
{\bf PI:} James F. Booth (jfbooth@ccny.cuny.edu), CUNY


\section*{Open Source Copyright Agreement}
\noindent
This package is distributed under the LGPLv3 license (see LICENSE.txt).

\section*{Functionality}
\noindent
The necessary scripts for this POD is given in mdtf/MDTF\_\$ver/var\_code/eulerian\_storm\_track

\hfill

eulerian\_storm\_track\_util.py is the code that computes the statistics. 

eulerian\_storm\_track\.py is the main driver code. 

plotter.py is the code used to create the plots. 

eulerian\_storm\_track\_obs.py is an internal code used to preprocess the observations and convert them to NetCDF files. 

\section*{Required programming languages and libraries}

\noindent
This package is written in Python 2, and requires the following Python packages: 

\indent
  os, Dataset, numpy, matplotlib, basemap \& netcdf4


\section*{Required model output variables}
\noindent
The following 3D (time, lat, lon) model fields are required:

\begin{itemize}
  \item V850 (units: m/s, daily)
\end{itemize}


\section*{References}
\noindent
Blackmon, M.L., 1976: A climatological spectral study of the 500mb geopotential height of the Northern Hemisphere. J. Atmos. Sci., 33, 1607-1623.

\noindent
Wallace, J.M., G-H Lim, M. L Blackmon, 1988: Relationship between cyclone tracks, anticyclone tacks and baroclinic waveguides. J. Atmos. Sci., 45, 439-462.

\section*{More about this Diagnostic}

\end{document}
